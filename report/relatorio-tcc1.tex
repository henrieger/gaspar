\documentclass[12pt]{article}

\usepackage{sbc-template}
\usepackage{graphicx,url}
\usepackage[utf8]{inputenc}
\usepackage[brazil]{babel}

\sloppy

\title{Relatório TCC 1}
\author{Henrique Luiz Rieger\inst{1}}
\address{Departamento de Informática -- Universidade Federal do Paraná
  (UFPR)\\
  Curitiba -- PR -- Brasil
}

\begin{document} 

\maketitle

\begin{resumo} 
\end{resumo}

\begin{abstract}
\end{abstract}

\section{Introdução}
- Importância da filogenia para a ciência e, principalmente, paleontologia \\
- Quais são os principais problemas computacionais em se fazer filogenia \\
  - Alto custo computacional da solução exata \\
  - Não há como garantir a solução correta, toda tentativa é uma hipótese \\
- Descrição da ideia geral do trabalho
\subsection{Análises filogenéticas}
- O que é uma filogenia \\
- Quais são as etapas de uma busca por árvores \\
  - Algoritmos de busca \\
  - Algoritmos de avaliação \\
  - Consenso \\
  - Cálculo do suporte \\
\subsection{Algoritmos genéticos}
- O que são algoritmos genéticos \\
- Vantagens e desvantagens dos GA \\
- Como um algoritmo genético pode ser usado em filogenia \\

\section{Trabalhos anteriores}
- Trabalhos que já usaram GA ou ML para filogenia: \\
  - \cite{cotta2002inferring} \\
  - \cite{mo-phylogenetics} \\
  - \cite{mo-phynet} \\
  - \cite{azouri2021harnessing} \\
  - \cite{parallel-gaml} \\
  - \cite{garli} \\

\section{Softwares atuais}
- Falar sobre TNT, PAUP*, MO-Phylogenetics e outros

\section{Experimentos preliminares}
- Experimentos com dados de \cite{dicynodonts-cris}, \cite{bremer-support} e algum dos bancos maiores

\section{Resultados}

\section{Proposta}
- Algoritmo genético baseado majoritariamente em \cite{cotta2002inferring} e \cite{garli} para busca de árvores usando os critérios de MP e IWMP a partir de dados morfológicos.


% \begin{figure}[ht]
% \centering
% \includegraphics[width=.3\textwidth]{fig2.jpg}
% \caption{This figure is an example of a figure caption taking more than one
%   line and justified considering margins mentioned in Section~\ref{sec:figs}.}
% \label{fig:exampleFig2}
% \end{figure}

\bibliographystyle{sbc}
\bibliography{relatorio-tcc1}

\end{document}
